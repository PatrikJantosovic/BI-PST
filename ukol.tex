%%%%%%%%%%%%%%%%%%%%%%%%%%%%%%%%%%%%%%%%%%%%%%%%%%%%%%%%%%%%%%%%%%%%%%
% Edit the title below to update the display in My Documents
%\title{Project Report}
%
%%% Preamble
\documentclass[paper=a4, fontsize=12pt]{scrartcl}
\usepackage[T1]{fontenc}
\usepackage{fourier}

\usepackage[czech]{babel}
\usepackage[utf8]{inputenc}									
\usepackage[protrusion=true,expansion=true]{microtype}	
\usepackage{amsmath,amsfonts,amsthm} % Math packages
\usepackage[pdftex]{graphicx}
\usepackage{subfig}	
\usepackage{url}
\usepackage[table,xcdraw]{xcolor}
\usepackage{float}

%%% Custom sectioning
\usepackage{sectsty}
\allsectionsfont{\centering \normalfont\scshape}


%%% Custom headers/footers (fancyhdr package)
\usepackage{fancyhdr}
\pagestyle{fancyplain}
\fancyhead{}											% No page header
\fancyfoot[L]{}											% Empty 
\fancyfoot[C]{}											% Empty
\fancyfoot[R]{\thepage}									% Pagenumbering
\renewcommand{\headrulewidth}{0pt}			% Remove header underlines
\renewcommand{\footrulewidth}{0pt}				% Remove footer underlines
\setlength{\headheight}{13.6pt}


%%% Equation and float numbering
\numberwithin{equation}{section}		% Equationnumbering: section.eq#
\numberwithin{figure}{section}			% Figurenumbering: section.fig#
\numberwithin{table}{section}				% Tablenumbering: section.tab#


%%% Maketitle metadata
\newcommand{\horrule}[1]{\rule{\linewidth}{#1}} 	% Horizontal rule

\title{
		%\vspace{-1in} 	
		\usefont{OT1}{bch}{b}{n}
		\normalfont \normalsize \textsc{Fakulta informačních technológií ČVUT} \\ [25pt]
		\horrule{0.5pt} \\[0.4cm]
		\huge BI-PST Domácí úkol \\
		\horrule{2pt} \\[0.5cm]
}
\author{
		\normalfont 								\normalsize
        Patrik Jantošovič\\[-3pt]		\normalsize
        Tomáš Zvara\\[-3pt]		\normalsize
        Tomáš Janecký\\[-3pt]		\normalsize
        \today
}
\date{}


%%% Begin document
\begin{document}
\maketitle
\section{Parametry a datový soubor}
Reprezentant: Patrik Jantošovič\\
K = den narození = 16\\
L = počet písmen v příjmení = 10\\
M = ((K+L)*46)mod11 + 1 = 2\\
Výsledkem je tedy datový soubor: case0102, mzda dle pohlaví\\

\subsection{Vytvoření datového souboru}
Řešení úloh předpokladá úspěšnou instalaci knihovni Sleuth2 a vytvoření .csv souboru s příslušnými daty.\\
Postup uvedeme jednou na začátku, abychom jsme se neopakovali.

\begin{itemize}
	\item >> install.packages("Sleuth2")
		\begin{itemize}
		\item Instalace package Sleuth2 
		\end{itemize}
	\item >> library(Sleuth2)
		\begin{itemize}
		\item Načítaní package Sleuth2 
		\end{itemize}
	\item >> write.table(case0102,"C:/data.csv",row.names=F,sep=";",dec=",")
		\begin{itemize}
		\item Zápis dat do .csv souboru
		\end{itemize}
\end{itemize}

\section{Řešení úkolů}
\subsection{Úkol číslo 1}
(1b) Načtěte datový soubor a rozdělte sledovanou proměnnou na příslušné dvě pozorované skupiny. Data stručně popište. 
Pro každu skupinu zvlášť odhadněte střední hodnotu, rozptyl a medián příslušného rozdělení.

\begin{itemize}
	\item >> data<-read.table("C:/data.csv",header=TRUE,sep=";") %ako budeme dalej riesit vyber dat , dalsie ulohy
		\begin{itemize}
		\item Načteme data z připraveného souboru
		\end{itemize}
	\item >> female<-data[1:61,]
		\begin{itemize}
		\item Načítaní dat pro pozorovanou skupinu: Female
		\end{itemize}
	\item >> male<-data[62:93,]
		\begin{itemize}
		\item Načítaní dat pro pozorovanou skupinu: Male
		\end{itemize}
	\item >> female<-female[,1] %da sa zdrcnut do jedneho prikazu nech to nie je ako na fitwiky
		\begin{itemize}
		\item Odřiznutí sloupce s pohlavím pro pozorovanou skupinu: Female
		\end{itemize}
	\item >> male<-male[,1]
		\begin{itemize}
		\item Odřiznutí sloupce s pohlavím pro pozorovanou skupinu: Male
		\end{itemize}
	\item >> length(male)
		\begin{itemize}
		\item Velikost dat pro pozorovanou skupinu: Male
		\end{itemize}
	\item >> length(female)
		\begin{itemize}
		\item Velikost dat pro pozorovanou skupinu: Female
		\end{itemize}
	\item >> var(male)
		\begin{itemize}
		\item Rozptyl pro pozorovanou skupinu: Male
		\end{itemize}
	\item >> var(female)
		\begin{itemize}
		\item Rozptyl pro pozorovanou skupinu: Female
		\end{itemize}
	\item >> mean(male)
		\begin{itemize}
		\item Střední hodnota pro pozorovanou skupinu: Male
		\end{itemize}
	\item >> mean(female)
		\begin{itemize}
		\item Střední hodnota pro pozorovanou skupinu: Female
		\end{itemize}
	\item >> median(male)
		\begin{itemize}
		\item Medián pro pozorovanou skupinu: Male
		\end{itemize}
	\item >> median(female)
		\begin{itemize}
		\item Medián pro pozorovanou skupinu: Female
		\end{itemize}
\end{itemize}
Výsledky zapíšeme do následujíci tabulky:\\
\begin{table}[htb]
\begin{tabular}{lllll}
\rowcolor[HTML]{EFEFEF}
Pohlaví & Velikost dat & Střední hodnota & Rozptyl & Medián \\
\cellcolor[HTML]{EFEFEF}Male     &     32       &         5956.875        &   477112.5      &   6000     \\
\cellcolor[HTML]{EFEFEF}Female   &      61       &         5138.852        &    291460.3     &   5220    
\end{tabular}
\end{table}

\subsection{Úkol číslo 2}
(1b) Pro každou skupinu zvlášť odhadněte hustotu a distribuční funkci pomocí histogramu a empirické distribuční funkce.

\begin{itemize}
	\item >> hist(female, freq=FALSE)
		\begin{itemize}
		\item Vykreslení histogramu female. freq=FALSE používame jako přepínač pro hustotu
		\end{itemize}
	\item >>hist(male, freq=FALSE)
		\begin{itemize}
	        \item Vykreslení histogramu female. freq=FALSE používame jako přepínač pro hustotu
		\end{itemize}
	\item >>plot(density(male))
		\begin{itemize}
		\item Vykreslení hustoty Male
		\end{itemize}
	\item >>plot(density(female))
		\begin{itemize}
		\item Vykreslení hustoty Female
		\end{itemize}
	\item >>plot(ecdf(male))
		\begin{itemize}
		\item Vykreslení empirické distribuční funkce pro Male
		\end{itemize}
	\item >>plot(ecdf(female))
		\begin{itemize}
		\item Vykreslení empirické distribuční funkce pro Female
		\end{itemize}
\end{itemize}
Výsledkem jsou grafy přiložené na následující stránce.%
\begin{figure}[H]%
    \centering
    \subfloat[Male]{{\includegraphics[width=0.4\textwidth]{hustota-male.png} }}%
    \qquad
    \subfloat[Female]{{\includegraphics[width=0.4\textwidth]{hustota-female.png} }}%
    \caption{Hustota}%
    \label{fig:example}%
\end{figure}
\begin{figure}[H]%
    \centering
    \subfloat[Male]{{\includegraphics[width=0.4\textwidth]{histogram-male.png} }}%
    \qquad
    \subfloat[Female]{{\includegraphics[width=0.4\textwidth]{histogram-female.png} }}%
    \caption{Histogram}%
    \label{fig:example}%
\end{figure}
\begin{figure}[H]%
    \centering
    \subfloat[Male]{{\includegraphics[width=0.4\textwidth]{ecdf-male.png} }}%
    \qquad
    \subfloat[Female]{{\includegraphics[width=0.4\textwidth]{ecdf-female.png} }}%
    \caption{Empirická distribuční funkce}%
    \label{fig:example}%
\end{figure}

\subsection{Úkol číslo 3}
(3b) Pro každou skupinu zvlášť najděte nejbližší rozdělení: Odhadněte parametry normálního, exponenciálního a rovnoměrného rozdělení. 
Zaneste příslušné hustoty s odhadnutými parametry do grafů histogramu. Diskutujte, které z rozdělení odpovídá pozorovaným datům nejlépe.
\begin{itemize}
	\item Male
		\begin{itemize}
		\item  >> hist(male, freq=FALSE)
		        \begin{itemize}
		        \item Porovnávame distribuční funkce různých rozdělení na histogramu.
		        \end{itemize}
		\item Normální rozdělení
			\begin{itemize}
		        \item >> maleV<-seq(min(male),max(male),10)
		                \begin{itemize}
		                \item vytvoříme si sekvenci hodnot od nejmenší po největší hodnoty 
		                \end{itemize}
		        \item >> maleNorm<-dnorm(maleV, mean = mean(male), sd = sd(male))
		                \begin{itemize}
		                \item využijeme funkci dnorm na převod pro body normálního rozdělení
		                \end{itemize}
		        \item >> lines(maleV,maleNorm, col=''blue'')
		                \begin{itemize}
		                \item vykreslíme normální rozdělení na histogram
		                \end{itemize}
		        \end{itemize}
		\item Exponenciální rozdělení
			\begin{itemize}
		        \item >> lambdaMale<-1/mean(male)
		                \begin{itemize}
		                \item vypočteme si parametr pro exponenciální rozdělení jako $\frac{1}{střední \ hodnota}$
		                \end{itemize}
		        \item >> maleExp<-dexp(maleV, lambdaMale)
		                \begin{itemize}
		                \item využijeme funkci dexp na vypočet bodu exponenciálního rozdělení
		                \end{itemize}
		        \item >>  lines(maleV,maleExp, col=''red'')
		                \begin{itemize}
		                \item vykreslíme exponencionální rozdělení na histogram
		                \end{itemize}
		        \end{itemize}
		\item Uniformní rozdělení
			\begin{itemize}
		        \item >> aMale<-mean(male)-sqrt(3*var(male))
		        \item >> bMale<-sqrt(3*var(male))+mean(male)
		                \begin{itemize}
		                \item vypočteme si parametr `a` a `b` pro uniformní rozdělení podle vztahu k střední hodnotě a rozptylu ze cvičení
		                \end{itemize}
		        \item >> maleUnif<-dunif(maleV, aMale,bMale)
		                \begin{itemize}
		                \item využijeme funkci dunif na výpočet bodu uniformního rozdělení
		                \end{itemize}
		        \item >> lines(maleV,maleUnif, col=''yellow'')
		                \begin{itemize}
		                \item vykreslíme uniformní rozdělení na histogram
		                \end{itemize}
		        \end{itemize}
		\end{itemize}
	\item Female
		\begin{itemize}
		\item  >> hist(female, freq=FALSE)
		        \begin{itemize}
		        \item Porovnávame distribuční funkce různých rozdělení na histogramu.
		        \end{itemize}
		\item Normálni rozdělení
			\begin{itemize}
		        \item >> femaleV<-seq(min(female),max(female),10)
		                \begin{itemize}
		                \item vytvoříme si sekvenci hodnot od nejmenší po největší hodnoty 
		                \end{itemize}
		        \item >> femaleNorm<-dnorm(femaleV, mean = mean(female), sd = sd(female))
		                \begin{itemize}
		                \item využijeme funkci dnorm na převod pro body normálního rozdělení
		                \end{itemize}
		        \item >> lines(femaleV,femaleNorm, col=''blue'')
		                \begin{itemize}
		                \item vykreslíme normální rozdělení na histogram
		                \end{itemize}
		        \end{itemize}
		\item Exponenciální rozdělení
			\begin{itemize}
		        \item >> lambdaFemale<-1/mean(female)
		                \begin{itemize}
		                \item vypočteme si parametr pro exponenciální rozdělení jako $\frac{1}{střední \ hodnota}$
		                \end{itemize}
		        \item >> femaleExp<-dexp(femaleV, lambdaFemale)
		                \begin{itemize}
		                \item využijeme funkci dexp na výpočet bodu exponenciálního rozdělení
		                \end{itemize}
		        \item >> lines(femaleV,femaleExp, col=''red'')
		                \begin{itemize}
		                \item vykreslíme exponencionální rozdělení na histogram
		                \end{itemize}
		        \end{itemize}
		\item Uniformní rozdělení
			\begin{itemize}
		        \item >> aFemale<-mean(female)-sqrt(3*var(female))
		        \item >> bFemale<-sqrt(3*var(female))+mean(female)
		                \begin{itemize}
		                \item vypočteme si parametr `a` a `b` pro uniformní rozdělení podle vztahu ke střední hodnotě a rozptylu ze cvičení
		                \end{itemize}
		        \item >> femaleUnif<-dunif(femaleV, aFemale,bFemale)
		                \begin{itemize}
		                \item využijeme funkci dunif na výpočet bodu uniformního rozdělení
		                \end{itemize}
		        \item >> lines(femaleV,femaleUnif, col=''yellow'')
		                \begin{itemize}
		                \item vykreslíme uniformní rozdělení na histogram
		                \end{itemize}
		        \end{itemize}
		\end{itemize}
\end{itemize}
Výsledkem jsou následujíci grafy:%
\begin{figure}[H]%
    \centering
    \subfloat[Male]{{\includegraphics[width=0.4\textwidth]{porovnani-male.png} }}%
    \qquad
    \subfloat[Female]{{\includegraphics[width=0.4\textwidth]{porovnani-female.png} }}%
    \caption{Porovnání rozdělení}%
    \label{fig:example}%
\end{figure}
Došli jsme k závěru, že se u obou datasetů jedná o Normální rozdělení.

\subsection{Úkol číslo 4}
(1b) Pro každou skupinu zvlášť vygenerujte náhodný výběr o 100 hodnotách z rozdělení, které jste zvolili jako nejbližší, 
s parametry odhadnutými v předchozím bodě. Porovnejte histogram simulovaných hodnot s pozorovanými daty.
\begin{itemize}
	\item Male
		\begin{itemize}
			\item >> maleRand100 = rnorm(100, mean(male), sd(male))
				\begin{itemize}
				\item vybereme 100 náhodných hodnot použitím funkce rnorm
				\end{itemize}
			\item >> hist(maleRand100)
				\begin{itemize}
				\item vykreslíme z náhodně vybraných dat histogram
				\end{itemize}
		\end{itemize}
	\item Female
		\begin{itemize}
			\item >> femaleRand100 = rnorm(100, mean(female), sd(female))
				\begin{itemize}
				\item vybereme 100 náhodných hodnot použitím funkce rnorm
				\end{itemize}
			\item >> hist(femaleRand100)
				\begin{itemize}
				\item vykrelslíme z náhodně vybraných dat histogram 
				\end{itemize}
		\end{itemize}
\end{itemize}
\newpage
Výsledkem jsou následujíci grafy:
\begin{figure}[H]
  \centering
  \subfloat[Male]{\includegraphics[width=0.4\textwidth]{nahodene100-male.png}}
  \qquad
  \subfloat[Female]{\includegraphics[width=0.4\textwidth]{nahodene100-female.png}}
  \caption{Vygenerované histogramy}
\end{figure}
Došli jsme k závěru, že zatímco vygenrovaný graf pro male \iffalse fakt nevim jak to nazvat \fi se velmi liší od původního
histogramu což je způsobeno malým množstvím dat. U female \iffalse ¯\_(ツ)_/¯ \fi kde máme $\approx$ 2x více dat se histogramy
velmi podobají i přes relativně malé množství dat.
\newpage
\subsection{Úkol číslo 5}
(1b) Pro každou skupinu zvlášť spočítejte oboustranný 95\% konfidenční interval pro střední hodnotu.
\newline\par
Ze zadaných dat neumíme přesně určit hodnotu rozptylu, proto musíme použít odhad intervalu kde se uplatňuje Studentovo rozdělení. Když předpokladáme, že naše rozdělení je normální (tedy minimálne alespoň podobné normálnímu), výsledek bude spolehlivý i při menším počtu dat. 
\[ 
\bigg(\overline{X_n} - t_{\alpha/2,n-1}\frac{s_n}{\sqrt{n}},
 \overline{X_n} + t_{\alpha/2,n-1}\frac{s_n}{\sqrt{n}} \bigg) 
\] \par
Spolehlivost intervalu je 95\%, tedy \(1 - \alpha = 0.95 \Rightarrow \alpha/2 = 0.025\)
\begin{itemize}
	\item Male
	\begin{itemize}
		\item[>>]\textit{ maleN = length(male)}
		\item[>>]\textit{ maleMean = mean(male)}
		\begin{itemize}
			\item uděláme bodový odhad pro střední hodnotu	
		\end{itemize}
		\item[>>]\textit{ maleSd = sd(male)}
		\begin{itemize}
			\item uděláme bodový odhad pro výběrovou směrodatnou odchylku 
		\end{itemize}
		\item[>>]\textit{maleDown = maleMean - qt(1-0.025,df=maleN-1)*maleSd/sqrt(maleN)} 
		\item[>>]\textit{ maleUp = maleMean + qt(1-0.025,df=maleN-1)*maleSd/sqrt(maleN)}
		\begin{itemize}
			\item funkce qt(...) očekáva jako první parametr \(1 - \alpha\) (mírný rozdíl od zápisu studentova t-rozdělení v přednáškach)    
		\end{itemize}
	\end{itemize}
	
	\[
	 \textbf{(5707.839, 6205.911)}
	\]

	\item Female
	\begin{itemize}
	\item[>>]\textit{ femaleN = length(female)}
	\item[>>]\textit{ femaleMean = mean(female)}
	\begin{itemize}
		\item uděláme bodový odhad pro střední hodnotu	
	\end{itemize}
	\item[>>]\textit{ femaleSd = sd(female)}
	\begin{itemize}
		\item uděláme bodový odhad pro výběrovou směrodatnou odchylku 
	\end{itemize}
	\item[>>]\textit{femaleDown = femaleMean - qt(1-0.025,df=femaleN-1)*femaleSd/sqrt(femaleN)} 
	\item[>>]\textit{ femaleUp = femaleMean + qt(1-0.025,df=femaleN-1)*femaleSd/sqrt(femaleN)}
	\begin{itemize}
		\item funkce qt(...) očekáva jako první parametr \(1 - \alpha\) (mírný rozdíl od zápisu studentova t-rozdělení v přednáškach)    
	\end{itemize}
\end{itemize}

\[
\textbf{(5000.585, 5277.12)}
\]
\end{itemize}

\newpage
\subsection{Úkol číslo 6}
(1b) Pro každou skupinu zvlášť otestujte na hladině významnosti 5\% hypotézu, zda je střední hodnota rovná hodnotě K (parametr úlohy - 16),
proti oboustranné alternativě. Můžete použít buď výsledek z předešlého bodu, nebo výstup z příslušné vestavěné funkce vašeho softwaru.
\\\\
\(H_0\) : střední hodnota je rovná 16\\
\(H_A\) : střední hodnota není rovná 16\\

\begin{itemize}
	\item Male
	\begin{itemize}
		\item[>>]\textit{t.test(male,mu=16,alternative="two.sided")}\\\\
			data:  male\\
			t = 48.654, df = 31, p-value < 2.2e-16\\
			alternative hypothesis: true mean is not equal to 16\\
			95 percent confidence interval:\\
			5707.839 6205.911\\
			sample estimates:\\
			mean of x\\
			5956.875
	\end{itemize}
	\item Male
	\begin{itemize}
		\item[>>]\textit{t.test(female,mu=16,alternative="two.sided")}\\\\
			data:  female\\
			t = 74.112, df = 60, p-value < 2.2e-16\\
			alternative hypothesis: true mean is not equal to 16\\
			95 percent confidence interval:\\
			5000.585 5277.120\\
			sample estimates:\\
			mean of x\\
			5138.852\\
	\end{itemize}
\end{itemize}
\par
Je zřejmé, že testovaná hodnota K = 16 v konfidenčním intervalu neleží. Na hladině spolehlivosti 5\% zamítneme \(H_0\) v prospěch  \(H_A\) (platí pro obě sledované skupiny).



\newpage
\subsection{Úkol číslo 7}
(2b) Na hladině spolehlivosti 5\% otestujte, jestli mají pozorované skupiny stejnou střední hodnotu.
Typ testu a alternativy stanovte tak, aby vaše volba nejlépe korespondovala s povahou zkoumaného problému.
\begin{itemize}
	\item [>>] t.test(female, mu=mean(male), alternative="two.sided")\\
	\\
	One Sample t-test\\
	\\
	data:  female\\
	t = -11.834, df = 60, p-value < 2.2e-16\\
	alternative hypothesis: true mean is not equal to 5956.875\\
	95 percent confidence interval:\\
 	5000.585 5277.120\\
	sample estimates:\\
	mean of x\\
 	5138.852\\

	\item [>>] t.test(male, mu=mean(female), alternative="two.sided")\\
	\\
	One Sample t-test\\
	\\
	data:  male\\
	t = 6.6993, df = 31, p-value = 1.707e-07\\
	alternative hypothesis: true mean is not equal to 5138.852\\
	95 percent confidence interval:\\
	5707.839 6205.911\\
	sample estimates:\\
	mean of x\\
	5956.875\\

\end{itemize}
Na hladině spolehlivosti 5\% (která je implicitní hodnotou \textit{conf.level}) testujeme, zda střední hodnota male $\in \textbf{(5000.585 , 5277.120)}$  konfidenčního intervalu female a 
obráceně, zda  střední hodnota female $\in \textbf{(5707.839, 6205.911)}$  konfidenčního intervalu male. Vyšlo nám $p = 0.00000000000000022$ pro první a
$p =  0.0000001707$ pro druhou hypotézu,tudíž zamítneme obě hypotézy ve prospěch alternativní hypotézy (nerovnají se).


%%% End document
\end{document}
