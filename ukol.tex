%%%%%%%%%%%%%%%%%%%%%%%%%%%%%%%%%%%%%%%%%%%%%%%%%%%%%%%%%%%%%%%%%%%%%%
% Edit the title below to update the display in My Documents
%\title{Project Report}
%
%%% Preamble
\documentclass[paper=a4, fontsize=12pt]{scrartcl}
\usepackage[T1]{fontenc}
\usepackage{fourier}

\usepackage[czech]{babel}
\usepackage[utf8]{inputenc}									
\usepackage[protrusion=true,expansion=true]{microtype}	
\usepackage{amsmath,amsfonts,amsthm} % Math packages
\usepackage[pdftex]{graphicx}
\usepackage{subfig}	
\usepackage{url}
\usepackage[table,xcdraw]{xcolor}
\usepackage{float}

%%% Custom sectioning
\usepackage{sectsty}
\allsectionsfont{\centering \normalfont\scshape}


%%% Custom headers/footers (fancyhdr package)
\usepackage{fancyhdr}
\pagestyle{fancyplain}
\fancyhead{}											% No page header
\fancyfoot[L]{}											% Empty 
\fancyfoot[C]{}											% Empty
\fancyfoot[R]{\thepage}									% Pagenumbering
\renewcommand{\headrulewidth}{0pt}			% Remove header underlines
\renewcommand{\footrulewidth}{0pt}				% Remove footer underlines
\setlength{\headheight}{13.6pt}


%%% Equation and float numbering
\numberwithin{equation}{section}		% Equationnumbering: section.eq#
\numberwithin{figure}{section}			% Figurenumbering: section.fig#
\numberwithin{table}{section}				% Tablenumbering: section.tab#


%%% Maketitle metadata
\newcommand{\horrule}[1]{\rule{\linewidth}{#1}} 	% Horizontal rule

\title{
		%\vspace{-1in} 	
		\usefont{OT1}{bch}{b}{n}
		\normalfont \normalsize \textsc{Fakulta informačních technológií ČVUT} \\ [25pt]
		\horrule{0.5pt} \\[0.4cm]
		\huge BI-PST Domácí úkol \\
		\horrule{2pt} \\[0.5cm]
}
\author{
		\normalfont 								\normalsize
        Patrik Jantošovič\\[-3pt]		\normalsize
        Tomáš Zvara\\[-3pt]		\normalsize
        Tomáš Janecký\\[-3pt]		\normalsize
        \today
}
\date{}


%%% Begin document
\begin{document}
\maketitle
\section{Parametry a datový soubor}
Reprezentant: Patrik Jantošovič\\
K = den narození = 16\\
L = počet písmen v příjmení = 10\\
M = ((K+L)*46)mod11 + 1 = 2\\
Výsledkem je teda datový soubor: case0102, mzda dle pohlaví\\

\subsection{Vytvoření datového souboru}
Řešení úloh předpokladá úspěšnou instalaci knihovni Sleuth2 a vytvoření .csv souboru s příslušnými datami.\\
Postup uvedeme jednou na začátku aby jsme se neopakovali.

\begin{itemize}
	\item >> install.packages("Sleuth2")
		\begin{itemize}
		\item Instalace package Sleuth2 
		\end{itemize}
	\item >> library(Sleuth2)
		\begin{itemize}
		\item Načítaní package Sleuth2 
		\end{itemize}
	\item >> write.table(case0102,"C:/data.csv",row.names=F,sep=";",dec=",")
		\begin{itemize}
		\item Zápis dat do .csv souboru
		\end{itemize}
\end{itemize}

\section{Řešení úkolú}
\subsection{Úkol číslo 1}
(1b) Načtěte datový soubor a rozdělte sledovanou proměnnou na příslušné dvě pozorované skupiny. Data stručně popište. Pro každu skupinu zvlášť odhadněte střední hodnotu, rozptyl a medián příslušného rozdělení.

\begin{itemize}
	\item >> data<-read.table("C:/data.csv",header=TRUE,sep=";")
		\begin{itemize}
		\item Načteme data z přiraveného souboru
		\end{itemize}
	\item >> female<-data[1:61,]
		\begin{itemize}
		\item Načítaní dat pro pozorovanou skupinu: Female
		\end{itemize}
	\item >> male<-data[62:93,]
		\begin{itemize}
		\item Načítaní dat pro pozorovanou skupinu: Male
		\end{itemize}
	\item >> female<-female[,1]
		\begin{itemize}
		\item Odseknutí sloupce s pohlavím pro pozorovanou skupinu: Female
		\end{itemize}
	\item >> male<-male[,1]
		\begin{itemize}
		\item Odseknutí sloupce s pohlavím pro pozorovanou skupinu: Male
		\end{itemize}
	\item >> length(male)
		\begin{itemize}
		\item Velkost dat pro pozorovanou skupinu: Male
		\end{itemize}
	\item >> length(female)
		\begin{itemize}
		\item Velkost dat pro pozorovanou skupinu: Female
		\end{itemize}
	\item >> var(male)
		\begin{itemize}
		\item Rozptyl pro pozorovanou skupinu: Male
		\end{itemize}
	\item >> var(female)
		\begin{itemize}
		\item Rozptyl pro pozorovanou skupinu: Female
		\end{itemize}
	\item >> mean(male)
		\begin{itemize}
		\item Střední hodnota pro pozorovanou skupinu: Male
		\end{itemize}
	\item >> mean(female)
		\begin{itemize}
		\item Střední hodnota pro pozorovanou skupinu: Female
		\end{itemize}
	\item >> median(male)
		\begin{itemize}
		\item Medián pro pozorovanou skupinu: Male
		\end{itemize}
	\item >> median(female)
		\begin{itemize}
		\item Medián pro pozorovanou skupinu: Female
		\end{itemize}
\end{itemize}
Výsledky zapíšeme do následujíci tabulky:\\
\begin{table}[htb]
\begin{tabular}{lllll}
\rowcolor[HTML]{EFEFEF}
Pohlavie & Velkost dat & Střední hodnota & Rozptyl & Medián \\
\cellcolor[HTML]{EFEFEF}Male     &     32       &         5956.875        &   477112.5      &   6000     \\
\cellcolor[HTML]{EFEFEF}Female   &      61       &         5138.852        &    291460.3     &   5220    
\end{tabular}
\end{table}

\subsection{Úkol číslo 2}
(1b) Pro každou skupinu zvlášť odhadněte hustotu a distribuční funkci pomocí histogramu a empirické distribuční funkce.

\begin{itemize}
	\item >> hist(female, freq=FALSE)
		\begin{itemize}
		\item Vykreslení histogramu female. freq=FALSE používame jako přepínač pro hustotu
		\end{itemize}
	\item >>hist(male, freq=FALSE)
		\begin{itemize}
	        \item Vykreslení histogramu female. freq=FALSE používame jako přepínač pro hustotu
		\end{itemize}
	\item >>plot(density(male))
		\begin{itemize}
		\item Vykreslení hustoty Male
		\end{itemize}
	\item >>plot(density(female))
		\begin{itemize}
		\item Vykreslení hustoty Female
		\end{itemize}
	\item >>plot(ecdf(male))
		\begin{itemize}
		\item Vykreslení empirický distribuční funkce pro Male
		\end{itemize}
	\item >>plot(ecdf(female))
		\begin{itemize}
		\item Vykreslení empirický distribuční funkce pro Female
		\end{itemize}
\end{itemize}
Výsledkem jsou grafy přiložené na následujíci stránce.%
\begin{figure}[H]%
    \centering
    \subfloat[lMale]{{\includegraphics[width=5cm]{hustota-male.png} }}%
    \qquad
    \subfloat[Female]{{\includegraphics[width=5cm]{hustota-female.png} }}%
    \caption{Hustota}%
    \label{fig:example}%
\end{figure}
\begin{figure}[H]%
    \centering
    \subfloat[Male]{{\includegraphics[width=5cm]{histogram-male.png} }}%
    \qquad
    \subfloat[Female]{{\includegraphics[width=5cm]{histogram-female.png} }}%
    \caption{Histogram}%
    \label{fig:example}%
\end{figure}
\begin{figure}[H]%
    \centering
    \subfloat[Male]{{\includegraphics[width=5cm]{ecdf-male.png} }}%
    \qquad
    \subfloat[Female]{{\includegraphics[width=5cm]{ecdf-female.png} }}%
    \caption{Empirická distribuční funkce}%
    \label{fig:example}%
\end{figure}

\subsection{Úkol číslo 3}
(3b) Pro každou skupinu zvlášť najděte nejbližší rozdělení: Odhadněte parametry normálního, exponenciálního a rovnoměrného rozdělení. Zaneste příslušné hustoty s odhadnutými parametry do grafů histogramu. Diskutujte, které z rozdělení odpovídá pozorovaným datům nejlépe.
\begin{itemize}
	\item Male
		\begin{itemize}
		\item  >> hist(male, freq=FALSE)
		        \begin{itemize}
		        \item Porovnávame distribuční funkce rúznich rozdelení na histogramu.
		        \end{itemize}
		\item Normálni rozdelení
			\begin{itemize}
		        \item >> maleV<-seq(min(male),max(male),10)
		                \begin{itemize}
		                \item vytvoríme si sekvenciu hodnot od nejmenší po nejvetší hodnoty 
		                \end{itemize}
		        \item >> maleNorm<-dnorm(maleV, mean = mean(male), sd = sd(male))
		                \begin{itemize}
		                \item využijeme funkci dnorm na převod pro body normálniho rozdelení
		                \end{itemize}
		        \item >> lines(maleV,maleNorm, col=''blue'')
		                \begin{itemize}
		                \item vykreslíme normálni rozdelení na histogram
		                \end{itemize}
		        \end{itemize}
		\item Exponenciálni rozdelení
			\begin{itemize}
		        \item >> lambdaMale<-1/mean(male)
		                \begin{itemize}
		                \item vypočteme si parametr pro exponenciálni rozdeleni jako 1 / střední hodnota
		                \end{itemize}
		        \item >> maleExp<-dexp(maleV, lambdaMale)
		                \begin{itemize}
		                \item využijeme funkci dexp na vypočet bodu exponenciálniho rozdelení
		                \end{itemize}
		        \item >>  lines(maleV,maleExp, col=''red'')
		                \begin{itemize}
		                \item vykreslíme exponencionálni rozdelení na histogram
		                \end{itemize}
		        \end{itemize}
		\item Uniformní rozdelení
			\begin{itemize}
		        \item >> aMale<-mean(male)-sqrt(3*var(male))
		        \item >> bMale<-sqrt(3*var(male))+mean(male)
		                \begin{itemize}
		                \item vypočteme si parametr `a` a `b` pro uniformní rozdeleni podle vztahu k střední hodnote a rozptylu ze cvičení
		                \end{itemize}
		        \item >> maleUnif<-dunif(maleV, aMale,bMale)
		                \begin{itemize}
		                \item využijeme funkci dunif na vypočet bodu uniformního rozdelení
		                \end{itemize}
		        \item >> lines(maleV,maleUnif, col=''yellow'')
		                \begin{itemize}
		                \item vykreslíme uniformni rozdelení na histogram
		                \end{itemize}
		        \end{itemize}
		\end{itemize}
	\item Female
		\begin{itemize}
		\item  >> hist(female, freq=FALSE)
		        \begin{itemize}
		        \item Porovnávame distribuční funkce rúznich rozdelení na histogramu.
		        \end{itemize}
		\item Normálni rozdelení
			\begin{itemize}
		        \item >> femaleV<-seq(min(female),max(female),10)
		                \begin{itemize}
		                \item vytvoríme si sekvenciu hodnot od nejmenší po nejvetší hodnoty 
		                \end{itemize}
		        \item >> femaleNorm<-dnorm(femaleV, mean = mean(female), sd = sd(female))
		                \begin{itemize}
		                \item využijeme funkci dnorm na převod pro body normálniho rozdelení
		                \end{itemize}
		        \item >> lines(femaleV,femaleNorm, col=''blue'')
		                \begin{itemize}
		                \item vykreslíme normálni rozdelení na histogram
		                \end{itemize}
		        \end{itemize}
		\item Exponenciálni rozdelení
			\begin{itemize}
		        \item >> lambdaFemale<-1/mean(female)
		                \begin{itemize}
		                \item vypočteme si parametr pro exponenciálni rozdeleni jako 1 / střední hodnota
		                \end{itemize}
		        \item >> femaleExp<-dexp(femaleV, lambdaFemale)
		                \begin{itemize}
		                \item využijeme funkci dexp na vypočet bodu exponenciálniho rozdelení
		                \end{itemize}
		        \item >> lines(femaleV,femaleExp, col=''red'')
		                \begin{itemize}
		                \item vykreslíme exponencionálni rozdelení na histogram
		                \end{itemize}
		        \end{itemize}
		\item Uniformní rozdelení
			\begin{itemize}
		        \item >> aFemale<-mean(female)-sqrt(3*var(female))
		        \item >> bFemale<-sqrt(3*var(female))+mean(female)
		                \begin{itemize}
		                \item vypočteme si parametr `a` a `b` pro uniformní rozdeleni podle vztahu k střední hodnote a rozptylu ze cvičení
		                \end{itemize}
		        \item >> femaleUnif<-dunif(femaleV, aFemale,bFemale)
		                \begin{itemize}
		                \item využijeme funkci dunif na vypočet bodu uniformního rozdelení
		                \end{itemize}
		        \item >> lines(femaleV,femaleUnif, col=''yellow'')
		                \begin{itemize}
		                \item vykreslíme uniformni rozdelení na histogram
		                \end{itemize}
		        \end{itemize}
		\end{itemize}
\end{itemize}
Výsledkem jsou následujíci grafy:%
\begin{figure}[H]%
    \centering
    \subfloat[Male]{{\includegraphics[width=5cm]{porovnani-male.png} }}%
    \qquad
    \subfloat[Female]{{\includegraphics[width=5cm]{porovnani-female.png} }}%
    \caption{Porovnání rozdelení}%
    \label{fig:example}%
\end{figure}
Došli sme k záveru že se u obou datasetu jedná o Normální rozdelení.

%%% End document
\end{document}